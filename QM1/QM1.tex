%!TEX program = xelatex
\documentclass[dvipsnames, svgnames,a4paper,11pt]{article}
% ----------------------------------------------------- 
%	加边框的命令
%	参考:https://tex.stackexchange.com/questions/531559/how-to-add-the-page-border-for-first-two-pages-in-latex
\usepackage{tikz}
\usetikzlibrary{calc}
\usepackage{eso-pic}
\AddToShipoutPictureBG{%
\begin{tikzpicture}[overlay,remember picture]
\draw[line width=0.6pt] % 边框粗细
    ($ (current page.north west) + (0.6cm,-0.6cm) $)
    rectangle
    ($ (current page.south east) + (-0.6cm,0.6cm) $); % 边框位置
\end{tikzpicture}}


\usepackage{xcolor}
\definecolor{c1}{HTML}{086173} % 目录颜色 原版为2752C9 紫灰色535AAA 蓝紫色0B0DB7 深蓝色070F94 湖绿色219394 松石灰绿086173
\definecolor{c2}{HTML}{E20129} % 引用颜色 原版\definecolor{c2}{RGB}{190,20,83} 橙色F24729

\usepackage{ctex}
\usepackage[top=28mm,bottom=28mm,left=15mm,right=15mm]{geometry}
\usepackage{hyperref} 
\hypersetup{
	colorlinks,
	linktoc = section, % 超链接位置,选项有section, page, all
	linkcolor = c1, % linkcolor 目录颜色
	citecolor = c1  % citecolor 引用颜色
}
\usepackage{amsmath,enumerate,multirow,float}
\usepackage{tabularx}
\usepackage{tabu}
\usepackage{subfig}
\usepackage{fancyhdr}
\usepackage{graphicx}
\usepackage{wrapfig}  
\usepackage{physics}
\usepackage{appendix}
\usepackage{amsfonts}

%
\usepackage{tcolorbox}
\tcbuselibrary{skins,breakable}
\newtcolorbox{tbox}[2][]{
    colframe=black!70!,
    breakable,
    enhanced,
	boxrule =0.5pt,
    title = {#2},
    fonttitle = \large\kaishu\bfseries,
	drop fuzzy shadow,
    #1
}
\newtcolorbox[auto counter,number within=section]{question}[1][]{
  top=2pt,bottom=2pt,arc=1mm,
  boxrule=0.5pt,
%   frame hidden,
  breakable,
  enhanced, %跨页后不会显示下边框
  coltitle=c1!80!gray,
  colframe=c1,
  colback=c1!3!white,
  drop fuzzy shadow,
  title={思考题~\thetcbcounter:\quad},
  fonttitle=\bfseries,
  attach title to upper,
  #1
}

% ---------------------------------------------------------------------
%	利用cleveref改变引用格式,\cref是引用命令
\usepackage{cleveref}
\crefformat{figure}{#2{\textcolor{c2}{Figure #1}}#3} % 图片的引用格式
\crefformat{equation}{#2{(\textcolor{c2}{#1})}#3} % 公式的引用格式
\crefformat{table}{#2{\textcolor{c2}{Table #1}}#3} % 表格的引用格式


% ---------------------------------------------------------------------
%	页眉页脚设置
\fancypagestyle{plain}{\pagestyle{fancy}}
\pagestyle{fancy}
\lhead{\kaishu 中山大学物理与天文学院基础物理实验\uppercase\expandafter{\romannumeral2}} % 左边页眉,学院 + 课程
\rhead{\kaishu 实验报告By黄罗琳} % 右边页眉,实验报告标题
\cfoot{\thepage} % 页脚,中间添加页码


% ---------------------------------------------------------------------
%	对目录、章节标题的设置
\renewcommand{\contentsname}{\centerline{\huge 目录}}
\usepackage{titlesec}
\usepackage{titletoc}
% \titleformat{章节}[形状]{格式}{标题序号}{序号与标题间距}{标题前命令}[标题后命令]
\titleformat{\section}{\centering\LARGE\songti}{}{1em}{}

% ---------------------------------------------------------------------
%   listing代码环境设置
\usepackage{listings}
\lstloadlanguages{python}
\lstdefinestyle{pythonstyle}{
backgroundcolor=\color{gray!5},
language=python,
frameround=tftt,
frame=shadowbox, 
keepspaces=true,
breaklines,
columns=spaceflexible,                   
basicstyle=\ttfamily\small, % 基本文本设置,字体为teletype,大小为scriptsize
keywordstyle=[1]\color{c1}\bfseries, 
keywordstyle=[2]\color{Red!70!black},   
stringstyle=\color{Purple},       
showstringspaces=false,
commentstyle=\ttfamily\scriptsize\color{green!40!black},%注释文本设置,字体为sf,大小为smaller
tabsize=2,
morekeywords={as},
morekeywords=[2]{np, plt, sp},
numbers=left, % 代码行数
numberstyle=\it\tiny\color{gray}, % 代码行数的数字字体设置
stepnumber=1,
rulesepcolor=\color{gray!30!white}
}




% ---------------------------------------------------------------------
%	其他设置
\def\degree{${}^{\circ}$} % 角度
\graphicspath{{./images/}} % 插入图片的相对路径
\allowdisplaybreaks[4]  %允许公式跨页 
\usepackage{lipsum}
\usepackage{adjustbox}
%\usepackage{mathrsfs} % 字体
%\captionsetup[figure]{name=Figure} % 图片形式
%\captionsetup[table]{name=Table} % 表格形式
\begin{document}
	
	
	
	
	% ---
	
	% 大标题
	\begin{center}
		\LARGE \textbf{Quantum Mechanics }  
		
		\LARGE \textbf{1. Quantum State and Hilbert Space}

		\LARGE \textbf{Solving Problems}
	\end{center}

	\section{Problem 1} 
	If the states $\{|1\rangle,|2\rangle,|3\rangle\}$ form an orthonormal basis and if the operator $\hat{K}$ has the properties

	$$\begin{aligned}&\hat{K}|1\rangle=2|1\rangle\\&\hat{K}|2\rangle=3|2\rangle\\&\hat{K}|3\rangle=-6|3\rangle\end{aligned}$$
	
	(a) Write an expression for $\hat{K}$ in terms of its eigenvalues and eigenvectors (projection operators).
	Use this expression to derive the matrix representing $\hat{K}$ in the $|1\rangle,|2\rangle,|3\rangle$ basis.
	
	(b) What is the expectation or average value of $\hat{K}$, defined as $\langle\alpha|\hat{K}|\alpha\rangle$, in the state
	
	$$|\alpha\rangle=\dfrac{1}{4}(-3|1\rangle+i\sqrt{6}|2\rangle+|3\rangle)$$

\subsection{a}

对于题目中给定的算符的性质可知:由于$\{|1\rangle,|2\rangle,|3\rangle\}$是一组正交归一基,即
$\langle1|1\rangle=1$,即可得如下形式:
\[
\hat{K} = 2|1\rangle\langle1| + 3|2\rangle\langle2| - 6|3\rangle\langle3|
\]

根据此形式即可得出$\hat{K}$ 在 $|1\rangle,|2\rangle,|3\rangle$ 基中的矩阵表示:
$$\begin{pmatrix}2&0&0\\0&3&0\\0&0&-6\end{pmatrix}$$


\subsection{b}

对于$|\alpha\rangle=\dfrac{1}{4}(-3|1\rangle+i\sqrt{6}|2\rangle+|3\rangle)$

可知其左矢为:$\frac{1}{4}((-3)^{\dagger} \langle1|+(i\sqrt{6})^{\dagger} \langle2|+\langle3|)$

根据$\{|1\rangle,|2\rangle,|3\rangle\}$为一组正交归一基,所以$\langle m|n \rangle=0$

所以可得
	$$\begin{aligned}
\langle\alpha| \hat{K} |\alpha\rangle & =\frac14\cdot\frac14((-3)^\dagger\left\langle1\right|+(i\sqrt{6})^\dagger\left\langle2\right|+\left\langle3\right|)\cdot\hat{K}\cdot(-3\left|1\right\rangle+i\sqrt{6}\left|2\right\rangle+\left|3\right\rangle) \\
&=\frac{1}{16}(9\left\langle1\right|\hat{K}\left|1\right\rangle+6\left\langle2\right|\hat{K}\left|2\right\rangle+\left\langle3\right|\hat{K}\left|3\right\rangle) \\
&=\frac{1}{16}(9\times2\left\langle1|1\right\rangle+6\times3\left\langle2|2\right\rangle-6\left\langle3|3\right\rangle) \\
&=\frac{15}{8}
\end{aligned}$$
	

\section{Problem 2} 

The Hamiltonian operator for a two-state system is given by

$$H=a(|1\rangle\langle1|-|2\rangle\langle2|+|1\rangle\langle2|+|2\rangle\langle1|),$$

where $a$ is a number with the dimension of energy. Find the energy eigenvalues and the corresponding
energy eigenkets (as linear combinations of $|1\rangle$ and $|2\rangle).$

解:对于问题中所给出的哈密顿算符有:

$$\hat{H}\begin{pmatrix}|1\rangle\\|2\rangle\end{pmatrix}=\begin{pmatrix}a\left|1\right\rangle+a\left|2\right\rangle\\a\left|1\right\rangle-a\left|2\right\rangle\end{pmatrix}=a\begin{pmatrix}1&1\\1&-1\end{pmatrix}\begin{pmatrix}|1\rangle\\|2\rangle\end{pmatrix}$$

所以此哈密顿算符可以表示为:

$$\hat{H}=a\begin{pmatrix}1&1\\1&-1\end{pmatrix}$$

根据线性代数可知,其本征值满足:

$$\det(\hat{H}-\lambda I)=0,$$

即:

$$\det\begin{pmatrix}a-\lambda&a\\a&-a-\lambda\end{pmatrix}=0.$$

$$(a-\lambda)(-a-\lambda)-a^2=0,$$

因此本征值为:

$$\lambda_1=\sqrt{2}a,\quad\lambda_2=-\sqrt{2}a.$$

对于$\lambda_1=\sqrt{2}a$

$$(\hat{H}-\lambda I) |v_1\rangle=\begin{pmatrix}a-\sqrt{2}a&a\\a&-a-\sqrt{2}a\end{pmatrix}\begin{pmatrix}c_1\\c_2\end{pmatrix}=0$$

解得:

$$c_1=(1+\sqrt{2})c_2$$

即:

$$|v_{1}\rangle==\frac{1}{\sqrt{4+2\sqrt{2}}}((1+\sqrt{2}) |1\rangle+|2\rangle)$$

同理可得:在$\lambda_2=-\sqrt{2}a$时,本征矢为:

$$\begin{aligned}|v_{2}\rangle=\frac{1}{\sqrt{4-2\sqrt{2}}}((1-\sqrt{2}) |1\rangle+|2\rangle)\end{aligned}$$

\section{Problem 3}

Consider the states $|\psi\rangle=9i\left|\phi_1\right\rangle+2\left|\phi_2\right\rangle$ and $\left|\chi\right\rangle=-\frac i{\sqrt{2}}\left|\phi_1\right\rangle+\frac1{\sqrt{2}}\left|\phi_2\right\rangle$ where the two vectors $\left|\phi_1\right\rangle$and $|\phi_{2}\rangle$ form a complete and orthonormal basis.

(a) Calculate the operators $|\psi\rangle\langle\chi|$ and $|\chi\rangle\langle\psi|.$ Are they equal?
item Find the Hermitian conjugates of $|\psi\rangle,|\chi\rangle,|\psi\rangle\langle\chi|$, and $|\chi\rangle\langle\psi|$.

( b)  Calculate Tr$( | \psi \rangle \langle \chi | )$ and Tr$(|\chi\rangle\langle\psi|).$ Are they equal?

(c) Calculate $|\psi\rangle\langle\psi|$ and $\chi\rangle\langle\chi|$ and the traces Tr$(|\psi\rangle\langle\psi|)$ and Tr$( | \chi \rangle \langle \chi | ) .$ Are they projection operators?

\subsection{a}$$\begin{aligned}
&\left|\psi\right\rangle\left\langle\chi\right| =(9i |\phi_{1}\rangle+2 |\phi_{2}\rangle)(\frac{i}{\sqrt{2}} \langle\phi_{1}|+\frac{1}{\sqrt{2}} \langle\phi_{2}|) \\
&=-\frac{9}{\sqrt{2}}\left|\phi_{1}\right\rangle\left\langle\phi_{1}\right|+\frac{9i}{\sqrt{2}}\left|\phi_{1}\right\rangle\left\langle\phi_{2}\right|+\frac{2i}{\sqrt{2}}\left|\phi_{2}\right\rangle\left\langle\phi_{1}\right|+\sqrt{2}\left|\phi_{2}\right\rangle\left\langle\phi_{2}\right| \\
&|\chi\rangle \langle\psi| =(-\frac{i}{\sqrt{2}} |\phi_{1}\rangle+\frac{1}{\sqrt{2}} |\phi_{2}\rangle)(-9i \langle\phi_{1}|+2 \langle\phi_{2}|) \\
&=-\frac{9}{\sqrt{2}}\left|\phi_{1}\right\rangle\left\langle\phi_{1}\right|-\frac{2i}{\sqrt{2}}\left|\phi_{1}\right\rangle\left\langle\phi_{2}\right|-\frac{9i}{\sqrt{2}}\left|\phi_{2}\right\rangle\left\langle\phi_{1}\right|+\sqrt{2}\left|\phi_{2}\right\rangle\left\langle\phi_{2}\right|
\end{aligned}$$

二者并不相等。

\subsection{b}

$$\mathrm{TR}(|\psi\rangle \langle\chi|)=-\frac{9}{\sqrt{2}}+\sqrt{2}=-\frac{7}{\sqrt{2}}\quad \mathrm{TR}(|\chi\rangle \langle\psi|)=-\frac{9}{\sqrt{2}}+\sqrt{2}=-\frac{7}{\sqrt{2}}$$

二者相等。

\subsection{c}
$$\begin{aligned}
|\psi\rangle \langle\psi|& =(9i\left|\phi_1\right\rangle+2\left|\phi_2\right\rangle)(-9i\left\langle\phi_1\right|+2\left\langle\phi_2\right|) \\
&=81\left|\phi_1\right\rangle\left\langle\phi_1\right|+18i\left|\phi_1\right\rangle\left\langle\phi_2\right|-18i\left|\phi_2\right\rangle\left\langle\phi_1\right|+4\left|\phi_2\right\rangle\left\langle\phi_2\right| \\
&=\begin{pmatrix}81&18i\\-18i&4\end{pmatrix} \\
|\chi\rangle \langle\chi|& =(-\frac{i}{\sqrt{2}} |\phi_{1}\rangle+\frac{1}{\sqrt{2}} |\phi_{2}\rangle)(\frac{i}{\sqrt{2}} \langle\phi_{1}|+\frac{1}{\sqrt{2}} \langle\phi_{2}|) \\
&=\frac{1}{2} |\phi_{1}\rangle \langle\phi_{1}|-\frac{i}{2} |\phi_{1}\rangle \langle\phi_{2}|+\frac{i}{2} |\phi_{2}\rangle \langle\phi_{1}|+\frac{1}{2} |\phi_{2}\rangle \langle\phi_{2} \\
&=\frac{1}{2}\begin{pmatrix}1&-i\\i&1\end{pmatrix}
\end{aligned}$$

对于投影算符来说,其为厄米算符且具备幂等性。

对于二者来说:$|\psi\rangle\langle\psi|$不满足第二条性质,所以不是投影算符。而$|\chi\rangle\langle\chi|$ 是一个投影算符。

\section{Problem 4}

\end{document}